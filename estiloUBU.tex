%
%Plantilla de estilo para la Universidad de Burgos.

% Paquetes para fuentes
\RequirePackage{times}
\RequirePackage[T1]{fontenc}

% Paquetes adicionales necesarios.
\RequirePackage{algorithm}
\RequirePackage{algorithmic}
\RequirePackage{booktabs}
\RequirePackage[table]{xcolor}
\RequirePackage{amsmath}
\RequirePackage{ifthen}
\RequirePackage{xtab}
\RequirePackage{multirow}
\RequirePackage{eurosym}
\RequirePackage{ulem}
\RequirePackage{morefloats}
%\RequirePackage{slashbox}

%
% Comentar este paquete para la versi�n impresa.
\usepackage[colorlinks,hypertexnames=false]{hyperref}


%
%Interlineado entre p�rrafos
\setlength{\parskip}{8pt}

%
% Referencia para los algoritmos
\newcommand{\veralg}[1]{(ver algoritmo~\ref{alg:#1})}

%
% Referencia para los algoritmos
\newcommand{\vertabla}[1]{(ver tabla~\ref{tabla:#1})}

%
% Definimos el nivel de detalle del ͭndice: 
\setcounter{tocdepth}{2}

%
% Control de viudas y hu�rfanas
\clubpenalty=10000
\widowpenalty=10000

%
% Para las tablas
\newcommand{\otoprule}{\midrule [\heavyrulewidth]}

%
% Comandos especiales

% Enfatizar nombres
\newcommand{\ubuemph}[1]{\textit{#1}}

% Flecha en modo texto
\newcommand{\flechaD}{$\rightarrow$}

% Comandos para las plantillas de casos de uso (evitar duplicar c�digo).
\newcommand{\alta}[1]{\multicolumn{#1}{l}{Alta, \sout{Media, Baja}}}
\newcommand{\media}[1]{\multicolumn{#1}{l}{\sout{Alta,} Media, \sout{Baja}}}
\newcommand{\baja}[1]{\multicolumn{#1}{l}{\sout{Alta, Media,} Baja}}


% El comando \figura nos permite insertar figuras comodamente, y utilizando
% siempre el mismo formato. Los parametros son:
% 1 -> Porcentaje del ancho de p�gina que ocupar� la figura (de 0 a 1)
% 2 --> Fichero de la imagen
% 3 --> Texto a pie de imagen
% 4 --> Etiqueta (label) para referencias
% 5 --> Opciones que queramos pasarle al \includegraphics
% 6 --> Opciones de posicionamiento a pasarle a \begin{figure}
\newcommand{\figuraConPosicion}[6]{%
  \setlength{\anchoFloat}{#1\textwidth}%
  \addtolength{\anchoFloat}{-4\fboxsep}%
  \setlength{\anchoFigura}{\anchoFloat}%
  \begin{figure}[#6]
    \begin{center}%
      \Ovalbox{%
        \begin{minipage}{\anchoFloat}%
          \begin{center}%
            \includegraphics[width=\anchoFigura,#5]{#2}%
            \caption{#3}%
            \label{#4}%
          \end{center}%
        \end{minipage}
      }%
    \end{center}%
  \end{figure}%
}

%
% Comando para incluir im�genes en formato apaisado (sin marco).
\newcommand{\figuraApaisadaSinMarco}[5]{%
  \begin{figure}%
    \begin{center}%
    \includegraphics[angle=90,height=#1\textheight,#5]{#2}%
    \caption{#3}%
    \label{#4}%
    \end{center}%
  \end{figure}%
}

%
% Nuevo comando para tablas peque�as (menos de una p�gina).
\newcommand{\tablaSmall}[5]{%
 \begin{table}
  \begin{center}
   \rowcolors {2}{gray!35}{}
   \begin{tabular}{#2}
    \toprule
    #4
    \otoprule
    #5
    \bottomrule
   \end{tabular}
   \caption{#1}
   \label{tabla:#3}
  \end{center}
 \end{table}
}

%
% Nuevo comando para tablas peque�as (menos de una p�gina).
\newcommand{\tablaSmallSinColores}[5]{%
 \begin{table}
  \begin{center}
   \begin{tabular}{#2}
    \toprule
    #4
    \otoprule
    #5
    \bottomrule
   \end{tabular}
   \caption{#1}
   \label{tabla:#3}
  \end{center}
 \end{table}
}

%
% Nuevo comando para tablas grandes con cabecera y filas alternas coloreadas en gris.
\newcommand{\tabla}[6]{%
  \begin{center}
    \tablefirsthead{
      \toprule
      #5
      \otoprule
    }
    \tablehead{
      \multicolumn{#3}{l}{\small\sl contin�a desde la p�gina anterior}\\
      \toprule
      #5
      \otoprule
    }
    \tabletail{
      \hline
      \multicolumn{#3}{r}{\small\sl contin�a en la p�gina siguiente}\\
    }
    \tablelasttail{
      \hline
    }
    \bottomcaption{#1}
    \rowcolors {2}{gray!35}{}
    \begin{xtabular}{#2}
      #6
      \bottomrule
    \end{xtabular}
    \label{tabla:#4}
  \end{center}
}

%
% Nuevo comando para tablas grandes con cabecera.
\newcommand{\tablaSinColores}[6]{%
  \begin{center}
    \tablefirsthead{
      \toprule
      #5
      \otoprule
    }
    \tablehead{
      \multicolumn{#3}{l}{\small\sl contin�a desde la p�gina anterior}\\
      \toprule
      #5
      \otoprule
    }
    \tabletail{
      \hline
      \multicolumn{#3}{r}{\small\sl contin�a en la p�gina siguiente}\\
    }
    \tablelasttail{
      \hline
    }
    \bottomcaption{#1}
    \begin{xtabular}{#2}
      #6
      \bottomrule
    \end{xtabular}
    \label{tabla:#4}
  \end{center}
}

%
% Nuevo comando para tablas grandes sin cabecera.
\newcommand{\tablaSinCabecera}[5]{%
  \begin{center}
    \tablefirsthead{
      \toprule
    }
    \tablehead{
      \multicolumn{#3}{l}{\small\sl contin�a desde la p�gina anterior}\\
      \hline
    }
    \tabletail{
      \hline
      \multicolumn{#3}{r}{\small\sl contin�a en la p�gina siguiente}\\
    }
    \tablelasttail{
      \hline
    }
    \bottomcaption{#1}
  \begin{xtabular}{#2}
    #5
   \bottomrule
  \end{xtabular}
  \label{tabla:#4}
  \end{center}
}
