El objetivo de este proyecto es desarrollar una herramienta did�ctica para facilitar a los alumnos que tengan que cursar asignaturas en las que se explique algoritmos de b�squeda, el funcionamiento del algoritmo de b�squeda en espacio de estados A estrella.

Para facilitar el acceso a esta herramienta desde cualquier dispositivo que posea un navegador web moderno, se decidi� desarrollar esta herramienta en formato de p�gina web. Seg�n estudios de Cisco en 2012 hab�a 8.700 millones de dispositivos con conexi�n a internet \citeotras{dispositivos_conectados}, esto hace que el alcance de la herramienta a sus potenciales usuarios sea enorme.

Uno de nuestro principales objetivos era que la herramienta fuera muy visual y f�cil de entender, quer�amos que cada paso fuera comprendido de un solo vistazo a trav�s de gr�ficos sencillos y tablas.

\section{Objetivos T�cnicos}
T�cnicamente los objetivos del proyecto son dos: poner en pr�ctica las t�cnicas de desarrollo m�s usadas actualmente por los mejores desarrolladores y aprender a usar las herramientas m�s modernas y populares del desarrollo web.

Respecto a las t�cnicas de desarrollo, el objetivo es tener un buen dise�o software para poder enfrentarnos a la fase de mantenimiento de una manera m�s legible y sencilla as� como conseguir crear nuevas funcionalidades sin tener que modificar en gran medida c�digo antiguo. Los costes de mantenimiento pueden abarcar el 80\% de un proyecto de software por lo que hay que valorar un buen dise�o.

Respecto a las herramientas de desarrollo ten�amos varios objetivos. El principal era aprender un nuevo lenguaje de programaci�n din�mico como es Ruby, que sea diferente a Java que es el principal lenguaje de programaci�n usado en el m�ster y poder as� complementar los conocimientos adquiridos durante la realizaci�n del mismo. 

Tambi�n se pretende usar el lenguaje de scripting m�s popular del mundo, debido a que puede ser usado en los navegadores web, JavaScript. Este lenguaje ha sido usado en una de las asignaturas del m�ster y consideramos que ser�a bueno profundizar un poco m�s en su uso debido a su enorme popularidad.

Por �ltimo, y por seguir la l�nea innovadora a la hora de elegir las herramientas de desarrollo para el proyecto, nos decantamos por usar una base de datos que no usara el cl�sico esquema entidad relaci�n de las bases de datos relacionales y decidimos elegir una base de datos NoSQL como es MongoDB, que no requiere un modelo est�tico y estructura de datos para almacenar la informaci�n.
