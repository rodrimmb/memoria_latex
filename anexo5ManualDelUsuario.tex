\section{Introducci�n}
Una vez que siguiendo los pasos del anexo anterior tenemos la aplicaci�n instalada, podemos empezar a usarla. En este anexo explicaremos como se trabaja con la herramienta.

\section{Manual de usuario}

\subsection{Pagina principal}
Como se puede ver la pagina principal se divide en dos partes, en la parte superior podemos ver un dialogo con una peque�a explicaci�n del objetivo de la web y justo debajo esta la tabla que contiene las b�squedas que se han creado previamente. Si anteriormente no se hubiera creado ninguna b�squeda o se hubieran eliminado todas esta tabla estar�a vac�a y solo de ver�a el encabezado de la tabla.

% Incluir imagen principal

Justo encima de la tabla hay dos botones uno para recargar la tabla y otro para a�adir nuevas b�squedas a la tabla. Cuando pulsemos el bot�n para a�adir una b�squeda nos aparecer� una ventana modal que nos pedir� que le demos un nombre a la b�squeda. Una vez introducido el nombre podemos pulsar a�adir y veremos como la b�squeda se ha a�adido a la tabla.

% Imagen modal busquedas

Desde la pagina principal tambi�n podremos eliminar las b�squedas que no nos interesen. como se puede ver al final de cada fila de la tabla de b�squedas se puede ver un bot�n con una papelera, al pulsar este nos pedir� confirmaci�n para eliminar la b�squeda.

% Imagen boton eliminar
% Imagen confirmacion eliminacion

Podemos pulsar encima de una de las b�squedas y acceder a la zona de trabajo.

\subsection{Zona de trabajo}
Una vez pulsemos encima de una de las b�squedas para acceder a la zona de trabajo, vemos que en la pagina web hay dos zonas diferenciadas la de la izquierda y la de la derecha. En la parte izquierda tendremos la informaci�n en tablas y en la derecha ira el grafo.

% Imagen work space

En la parte superior izquierda vemos tres pesta�as. La primera pesta�a con el icono de una casa nos devolver� a la pagina inicial, la segunda pesta�a <<\en{Nodes Input}>>, que es la de por defecto, nos da informaci�n y opciones sobre los nodos y la ultima pesta�a <<\en{Solution}>> muestra informaci�n sobre como se resuelve el algoritmo y opciones propias para resolver la b�squeda.

% Imagen detalle pesta�as



